% LATEX TEMPLATE BY EMIL PERSSON
% Edited by Niklas Persson 2025
% Layout design: Bo Tonnquist, Baseline Management AB, 2018  
\documentclass[10pt]{projectdoc}
\fancyFoot{© ProjectBase 7.0 Baseline Management AB, 2018}{\today}
% REMOVE ALL AUTOMATIC INDENTS FROM PAPER
\setlength{\parindent}{0pt}
% \usepackage[style=ieee]{biblatex}
% \addbibresource{references.bib}
\title{Project plan}
\projectname{}
\clientname{}
\managername{}
\begin{document}
\maketitle
\thispagestyle{fancy}
\infotable

% Copy and Paste any table into https://www.tablesgenerator.com/
% File -> From latex code... | Edit the imported table and generate new

\section{Executive summary}
\helper{A short summary of the project plan}

\section{Background}
\helper{Description with a clear and defined connection to the goals. It is advisable
to connect to any related project in the background description.}
\subsection{Previous iterations (if applicable)}
What can you learn from previous iteration? What conclusion did they make? 
\subsection{Related work}
Can include research articles, industrial platforms and components, videos etc. Make sure you refer to the different sources in a proper way. Research articles are mandatory. 
\section{Purpose}
\helper{The impact the project is expected to create, i.e. why it is important to execute the project.  }
\section{Goal}
\helper{The result the project should deliver, i.e. what should be achieved when the project is executed.  }
\section{Scope}
\helper{What is included as part of the project and must be performed in order to deliver the goal. The scope is described with a
WBS at the overarching level – main packages with a brief description of each. The complete WBS should be included as
an attachment.}
\section{Limitations}
\helper{What the project should not deliver. The purpose is to avoid false expectations among the different stakeholders.}

\section{Requirement}
\subsection{Product requirements}
\helper{The product specification describes the product that is to be delivered. It is a description of the product in terms of its functionality, performance, quality, etc.}
\subsubsection{Functional requirements}
\helper{The functional requirements describe the functions that the product must do.}
\subsubsection{Non-functional requirements}
\helper{The non-functional requirements describe the technical specifications of the product in details: e.g. colors, performance, quality metrics.}

\subsection{Project requirements}
\helper{Requirements on the execution and prioritization between the project’s triple constraints.}
\subsection{Prerequisites}
\helper{Demands on the project’s sponsor/owner or client that have to be achieved to ensure the project’s execution and result.}


\section{Planning}
\subsection{Milestone plan}
\helper{Stakeholders wants an overarching flow chart or table of the project’s most important milestones as a indicator if the project is falling behind.}

\subsection{Work breakdown structure}
The Work Breakdown Structure (WBS) is a fundamental tool in project management, establishing a systematic and structured approach to break down the intricate project into manageable components. This detailed, hierarchical decomposition of the project's scope presents a visual or textual representation that facilitates an understanding of the project's structure, workflow, and tasks necessary for successful project completion. A key note is that the WBS should only include items related to the project and not the project course, this excludes items such as course assignments as they are not a part of the project, only the course.

If you need an anecdote, picture this: "\textit{You are a consultant assigned to help clients develop a Work Breakdown Structure (WBS) for their project. This WBS will outline the project's components, giving a clear roadmap of what is to come. You also have internal tasks like time reports, presentations, and assignments for your consulting firm. However, it is important to remember that these internal duties belong outside the client's WBS. They are part of your job, not the client's project. By keeping the WBS client-specific, you maintain clarity and ensure an effective project management strategy.}"

The WBS dissects the project into multiple layers for ease of management, separating the work into components, work packages, deliverables, and tasks.

\begin{itemize}
\item \textbf{Components} represent the broad sections or stages of the project.
\item \textbf{Work Packages} are subsets of these components, which can be further broken down into specific deliverables.
\item \textbf{Deliverables} are tangible or intangible goods or services produced in the project. They can be handed over physically or digitally to another person or team, forming the backbone of each work package and contributing substantially to the overall project's progression.
\item \textbf{Tasks} represent the smallest work units necessary to complete each deliverable, providing clear action items for project participants.
\end{itemize}

By enabling systematic project planning, effective resource allocation, and reliable progress tracking, the WBS becomes an invaluable aid in navigating the complexity of projects and driving them towards successful outcomes.

A small text-based example, recommend visual based, of applying a WBS for a mobile autonomous robot development project is illustrated below, and only one component is added to keep it short for the example. This example should provide a more tangible understanding of the WBS and its purpose.

\begin{table}[H]
    \centering
    \begin{tabular}{|l|l|l|}
    \hline
    \textbf{\textbf{ID}} & \textbf{Type} & \textbf{Activity}                                          \\ \hline
    1                    & Root          & Mobile Autonomous Robot Development                        \\ \hline
    1.1                  & Component     & Navigation and Path Planning                               \\ \hline
    1.1.1                & Work Package  & Sensor Integration                                         \\ \hline
    1.1.1.1              & Deliverable   & Sensor Selection Report                                    \\ \hline
    1.1.1.1.1            & Task          & Research Suitable Sensors (LiDAR, RADAR, Ultrasonic, etc.) \\ \hline
    1.1.1.1.2            & Task          & Prepare Sensor Selection Report                            \\ \hline
    1.1.1.2              & Deliverable   & Sensor Installation Manual                                 \\ \hline
    1.1.1.2.1            & Task          & Install Sensors on Robot Chassis                           \\ \hline
    1.1.1.2.2            & Task          & Prepare Sensor Installation Manual                         \\ \hline
    1.1.1.3              & Deliverable   & Sensor Data Fusion                                         \\ \hline
    1.1.1.3.1            & Task          & Develop Sensor Data Fusion Algorithm                       \\ \hline
    1.1.1.3.2            & Task          & Implement and Test Sensor Data Fusion Algorithm            \\ \hline
    1.1.2                & Work Package  & Path Planning                                              \\ \hline
    1.1.2.1              & Deliverable   & Path Planning                                              \\ \hline
    1.1.2.1.1            & Task          & Research Suitable Path Planning Algorithms                 \\ \hline
    1.1.2.1.2            & Task          & Implement Selected Path Planning Algorithm                 \\ \hline
    1.1.2.2              & Deliverable   & Path Planning Testing Report                               \\ \hline
    1.1.2.2.1            & Task          & Test Path Planning Algorithm in Simulated Environment      \\ \hline
    1.1.2.2.2            & Task          & Prepare Path Planning Testing Report                       \\ \hline
    \end{tabular}%
    \end{table}
\subsection{Schedule}
\helper{Activity plan with a time axis where duration and connection between activities and milestones are shown.}



\section{Project budget}
\helper{The project’s preliminary calculation – a  outline of internal and external costs for resources needed to execute the
project.}

\begin{table}[H]
    \begin{tabularx}{\columnwidth}{|X|X|X|X|}
        \hline
        Internal costs & External costs & Other costs & Summary \\ \hline
                       &                &             &         \\ \hline
                       &                &             &         \\ \hline
                       &                &             &         \\ \hline
\end{tabularx}
\end{table}

\section{Tools and standards}
We leverage the following tools, technologies, and systems in our project:

\begin{itemize}
\item \textbf{Software Development Tools:} Include various programming languages and 3D modelling software. For instance, [Programming Language] is used for [purpose] and can be downloaded from [source]. For 3D modelling, we use [Software name], which can be downloaded from [source].
\item \textbf{Project Management Tools:} We utilize [Tool/Platform Name, version] for [purpose], and it can be downloaded from [source].
\item \textbf{Communication Tools:} For effective communication, we use [Tool/Platform Name, version], which can be downloaded from [source].
\item \textbf{Hardware Tools and Software:} We utilize tools like a 3D printer ([specify model]) and PCB Design Software ([Software Name]) which can be downloaded from [source].
\item \textbf{Operating Systems:} We primarily operate on [Operating System name and version]. Please align your system with the same for consistency and compatibility. If your system operates differently, notify the team for assistance with potential compatibility issues.
\end{itemize}

Please install the correct version numbers as specified for consistency and compatibility.

\subsection{Coding Standards and Guidelines}

To ensure high code quality and readability, we follow these coding standards and guidelines: [Outline specific standards and guidelines]. For more comprehensive information, refer to our detailed coding standards document [footnote or ref to document].

\subsection{Version Control}

We utilize [Version Control System Name, e.g., Git] to manage changes to our project files and maintain different versions. For detailed guidance on using this system, please refer to our [footnote or ref to version control usage guide].

\section{Communication plan}
\helper{Describe how you plan to communicate with the project owner and how you plan to communicate among each other.}



\section{Documentation plan}
\helper{Plan for how the project’s documentation is to be managed. Example listed below.}

The purpose of a Documentation Plan is to provide a structured framework for creating, storing, reviewing, and disseminating project-related information. It ensures systematic recording and retention of all details, facilitating efficient communication, transparency, accountability, and continuity throughout the project. Comprehensive documentation is a critical tool in project management, aiding in tracking progress, making informed decisions, and providing references for future initiatives. Furthermore, it promotes knowledge sharing and learning during the project and beyond.

The following sections present a detailed example of a Documentation Plan. It outlines what needs to be documented, how and when to do so, who is responsible for various documentation tasks, where the documents will be stored, and the review and approval process for these documents. This example serves as a guide and should be tailored to fit your project's specific needs and circumstances.

\subsection{What to Document}

Define the types of documentation that will be required. This could include design documents, technical specifications, meeting minutes, code comments, testing reports, user manuals, troubleshooting guides, etc.

\subsection{How to Document}

Specify the format and structure of each document. For example, technical specifications might require diagrams and detailed descriptions, while meeting minutes may be more informal. Develop templates for consistency.

\subsection{When to Document}

Define when each document should be updated. For instance, design documents should be updated whenever a design decision is made, testing reports after every test run, and meeting minutes following each meeting.



\subsection{Where to Store Documents}

Define where documents will be stored. This could be on a shared drive, a project management tool, or a version control system. Ensure it is accessible to all team members and has appropriate backup and security measures.

\subsection{Document Review Process}

Specify a process for reviewing and approving documents. This could involve peer reviews or formal approval processes.

\subsection{Training}

Ensure everyone involved in the project is aware of the documentation plan and trained on how to use the tools and templates. This can be part of the initial project orientation or documentation training session.


\section{Handover plan}
\helper{How to deliver the product to the client and implement it into the environment it is meant for. The following subsections is an general simple example}

\noindent This Handover Plan outlines the systematic process of transferring all project deliverables, documentation, tools, and knowledge to the project owner at the end of the project or a project phase. The objective is to ensure a smooth transition, preserving all the hard work done on the project and providing a firm basis for future work. Understanding this plan will ensure that all team members know what is required in the final stages of the project.

\subsection{Handover Methodology}

Our handover methodology includes [describe methods and steps such as inventory check, final documentation, presentation, and sign-off]. These steps have been chosen to ensure a comprehensive and efficient handover process.

\subsection{Handover Tools, Technologies, and Systems}

We leverage the following tools, technologies, and systems during our handover process:

\begin{itemize}
\item \textbf{Inventory Management Tools:} We utilize [Tool/Platform Name, version] to catalogue and track all project deliverables for handover, which can be downloaded from [source].
\item \textbf{Presentation Tools:} If necessary, we use [Tool/Platform Name, version] to conduct presentations or demonstrations, and it can be downloaded from [source].
\end{itemize}

Please install the correct version numbers as specified for consistency and compatibility.


\subsection{Documentation}

Comprehensive project documentation, including user manuals, technical documentation, and project reports, will be provided during the handover process. These documents can be accessed from [footnote or ref to Document Storage/Management System].

\subsection{Presentation and Demonstration}

Our team will conduct presentations and demonstrations where necessary to familiarize the project owner with the project's operation and results.

\subsection{Final Sign-off}

The project owner will sign off, indicating successful handover and completion. 

\subsection{Version Control for Handover Artifacts}

We utilize [Version Control System Name, e.g., Git] to manage changes to our handover artefacts and maintain different versions. For detailed guidance on using this system, please refer to our [footnote or ref to version control usage guide].

\subsection{Follow-up and Feedback}

We highly value feedback from the project owner and have a process for follow-up [describe follow-up process] to ensure that all aspects of the handover were completed satisfactorily and address any potential concerns post-hand-over.

\section{Other}
\helper{Anything else?}

\newpage
\bibliographystyle{unsrtnat}
\bibliography{references}

% \printbibliography
% \appendices
\end{document}
